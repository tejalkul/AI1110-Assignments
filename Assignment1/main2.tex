\documentclass[12pt, twocolumn]{article}
\usepackage[utf8]{inputenc}
\usepackage{amsmath}
\usepackage[margin = 1.1in]{geometry}
\usepackage{graphicx}
\graphicspath{{images/}}
\setlength{\columnsep}{0.50cm}

\title{AI1110 Assignment 1}
\author{Tejal Kulkarni CS21BTECH11058}
\date{March 2022}

\begin{document}
\maketitle

%\section*{Q8(c)}
    %\includegraphics[scale=1.10]{question.png}
   % \section*{Solution:}
\textbf{Q8(c):} The printed price of an air conditioner is  Rs.45,000/-. The wholesaler allows a discount of 10\% to the shopkeeper. The shopkeeper sells the article to the customer at a discount of 5\% of the marked price. Sales tax (under VAT) is charged at the rate of 12\% at every stage. Find:
\begin{enumerate}
\item[(i)]VAT paid by the shopkeeper to the government.
\item[(ii)]The total amount paid by the customer inclusive of tax.\\ \\
\end{enumerate}
\textbf{Solution:}\\
Let MP be marked price,SP be selling price, d be the discount and disc be the discount amount.Then, \\
\begin{equation*}
 disc = MP\times\frac{d}{100} \hspace{1cm}...(1)\\ 
\end{equation*}

\begin{equation*}
    SP = MP - disc \hspace{1cm}...(2)\\ \\
\end{equation*}
Now let s be the sales tax and t be the tax amount. Therefore,\\
\begin{equation*}
    t = SP\times\frac{s}{100} \hspace{1cm}...(3)\\ \\ \\
\end{equation*}
Marked  Price = Rs.45,000/- \\
Sales  Tax = 12\% \\
Discount  for  shopkeeper = 10\% \\ \\
Discount  amount for shopkeeper =  \begin{equation*}
    45000\times\frac{10}{100} = Rs.4500 \hspace{1cm}... by \hspace{0.2cm} (1)\\ 
    \end{equation*} 
Selling Price for shopkeeper = \begin{equation*}
    45000 - 4500 = Rs.40500 \hspace{1cm}... by \hspace{0.2cm} (2)\\
\end{equation*}
Sales Tax for shopkeeper = \begin{equation*} \\
    40500\times\frac{12}{100} =  Rs.4860 \hspace{1cm}... by \hspace{0.2cm} (3)\\
\end{equation*}
Discount for customer = 5\% \\ \\
Discount amount for customer  = 
\begin{equation*} 
     45000\times\frac{5}{100} = Rs.2250 \hspace{1cm}... by \hspace{0.2cm} (1)\\
    \end{equation*}
Selling Price for customer = \begin{equation*}\\
    45000 - 2250 =  Rs.42750 \hspace{1cm}... by \hspace{0.2cm} (2)\\ 
\end{equation*}
Sales tax for customer = \begin{equation*}\\
    42750\times\frac{12}{100} = Rs.5130 \hspace{1cm}... by \hspace{0.2cm} (3)\\
\end{equation*} \\
Hence,\\
i) Tax for customer - Tax for shopkeeper = 
    VAT paid by shopkeeper to government  \\
     \begin{equation*}
        = 5130 - 4860 = \fbox{Rs.270}\\
   \end{equation*}
ii) Selling price for customer + Tax for customer = Total amount paid by customer\\ 
    \begin{equation*}
       = 42750 + 5130 = \fbox{Rs.47880}\\
   \end{equation*} \\ \\
The output of the program used for verification:\\ \\
\includegraphics[scale=1.1]{codeoutput.png} \\

\end{document}
