\documentclass[12pt, twocolumn]{article}
\usepackage[utf8]{inputenc}
\usepackage{amsmath}
\usepackage[margin = 1.1in]{geometry}
\usepackage{graphicx}
\graphicspath{{images/}}
\setlength{\columnsep}{0.75cm}
\usepackage{hyperref}
\hypersetup{
colorlinks=true,
    linkcolor= red,
    }

\title{AI1110 Assignment 1}
\author{Tejal Kulkarni CS21BTECH11058}
\date{March 2022}

\begin{document}
\maketitle

\textbf{Q8(c):} The printed price of an air conditioner is  Rs.45,000/-. The wholesaler allows a discount of 10\% to the shopkeeper. The shopkeeper sells the article to the customer at a discount of 5\% of the marked price. Sales tax (under VAT) is charged at the rate of 12\% at every stage. Find:
\begin{enumerate}
\item[(i)]VAT paid by the shopkeeper to the government.
\item[(ii)]The total amount paid by the customer inclusive of tax. 
\end{enumerate}
\textbf{Solution:}\\
According to the Table \ref{table:1}:\\
Formulae used:
\begin{align}
  \text{disc} &= \text{MP}\times\frac{\text{d}}{100}\hspace{1cm}...\\ 
  \text{SP} &= \text{MP} - \text{disc} \hspace{1cm}...\\ 
  \text{t} &= \text{SP}\times\frac{\text{s}}{100}\hspace{1cm}...
\end{align}
Marked  Price(MP) = Rs.45,000/- \\
Sales  Tax(s) = 12\% \\
Discount  for  shopkeeper (d1) = 10\% \\ \\
Discount  amount for shopkeeper (disc1) =  \begin{equation*}
    45000\times\frac{10}{100} = Rs.4500 \hspace{1cm}... by \hspace{0.2cm} (1) 
    \end{equation*} 
Selling Price for shopkeeper (SP1) = \begin{equation*}
    45000 - 4500 = Rs.40500 \hspace{1cm}... by \hspace{0.2cm} (2)
\end{equation*}
Sales Tax for shopkeeper(t1) = \begin{equation*} 
    40500\times\frac{12}{100} =  Rs.4860 \hspace{1cm}... by \hspace{0.2cm} (3)
\end{equation*}
Discount for customer (d2) = 5\% \\ \\
Discount amount for customer(disc2)  = 
\begin{equation*} 
     45000\times\frac{5}{100} = Rs.2250 \hspace{1cm}... by \hspace{0.2cm} (1)
    \end{equation*}
Selling Price for customer (SP2) = \begin{equation*}\\
    45000 - 2250 =  Rs.42750 \hspace{1cm}... by \hspace{0.2cm} (2) 
\end{equation*}
Sales tax for customer (t2) = \begin{equation*}\\
    42750\times\frac{12}{100} = Rs.5130 \hspace{1cm}... by \hspace{0.2cm} (3)
\end{equation*} \\
Hence,
\begin{enumerate}
\item[(i)]Tax for customer - Tax for shopkeeper = 
    VAT paid by shopkeeper to government(V)  \\
     \begin{equation*}
        = 5130 - 4860 = \fbox{Rs.270}\\
     \end{equation*}
\item[(ii)]Selling price for customer + Tax for customer = Total amount paid by customer(T)\\ 
    \begin{equation*}
       = 42750 + 5130 = \fbox{Rs.47880}\\
   \end{equation*} 
\end{enumerate}
 The output of the program used for verification:\\ \\
\includegraphics[scale=1.1]{codeoutput.png} 
\begin{table}[h]
   \caption{\textbf{Table with input and output variables, their symbols, their formulae and values:}}
    \label{table:1}
   \begin{center}
    \begin{tabular}{|c|c|c|c|}
       \hline
       \textbf{Description}  & \textbf{Symbol} &  \textbf{Formula}  &  \textbf{Value(obtained/given)} \\
       \hline
       \hline
       Marked Price & MP & - & Rs.45000 \\  
       \hline
       Discount for shopkeeper & d1 & - & 10\% \\
       \hline
       Discount amount for shopkeeper & disc1 & $\text{MP}\times\dfrac{d1}{100}$ & Rs.4500\\
       \hline
       Selling Price for shopkeeper & SP1 & $\text{MP} -  \text{disc1}$ & Rs.40500 \\
       \hline
       Sales Tax & s & - & 12\% \\
       \hline
       Tax amount for shopkeeper & t1 & $\text{SP1}\times\dfrac{s}{100}$ & Rs.4860\\
       \hline
       Discount for customer & d2 & - & 5\% \\
       \hline
       Discount amount for customer & disc2 & $\text{MP}\times\dfrac{d2}{100}$ & Rs.2250 \\
       \hline
       Selling Price for customer & SP2 & $\text{MP} - \text{disc2}$ & Rs.42750 \\
       \hline
       Tax amount for customer & t2 & $\text{SP2}\times\dfrac{s}{100}$ & Rs.5130\\
       \hline
       VAT paid by shopkeeper to govt. & V & $\text{t2} - \text{t1}$ & Rs.270\\
       \hline
       Total Amount paid by customer & T & $\text{SP2} + \text{t2}$ & Rs.47880\\
       \hline
       
    \end{tabular}
\end{center}

\end{table}
\end{document}
