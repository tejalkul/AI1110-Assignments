\documentclass[journal,12pt,two column]{IEEEtran}


\usepackage{listings}
\usepackage{amssymb}
\usepackage[cmex10]{amsmath}
\usepackage{amsthm}
\usepackage[export]{adjustbox}
\usepackage{bm}
\def\inputGnumericTable{} 

\usepackage[latin1]{inputenc}                                 
\usepackage{color}                                            
\usepackage{array}   
\usepackage{longtable}
\usepackage{enumitem}
\usepackage{calc}                                             
\usepackage{multirow}                                         
\usepackage{hhline}                                           
\usepackage{ifthen}  
\usepackage{mathtools}
\usepackage{tikz}
\usepackage{listings}
\usepackage{color}                                            %%
\usepackage{array}                                            %%
\usepackage{caption} 
\usepackage{graphicx}
\graphicspath{{images/}}
\newcommand*{\permcomb}[4][0mu]{{{}^{#3}\mkern#1#2_{#4}}}
\newcommand*{\perm}[1][-3mu]{\permcomb[#1]{P}}\usepackage{setspace}
%\captionsetup[table]{skip=3pt} 

\title{AI1110 Assignment 11 }
\author{Tejal Kulkarni \\ CS21BTECH11058 \\\vspace*{20pt} May 2022 \\  Papoullis Text Book }

\begin{document}
\maketitle

\newcommand{\solution}{\noindent \textbf{Solution: }}
\providecommand{\pr}[1]{\ensuremath{\Pr\left(#1\right)}}
\providecommand{\cdf}[2]{\ensuremath{\text{F}_{#1}\left(#2\right)}}
\providecommand{\qfunc}[1]{\ensuremath{Q\left(#1\right)}}
\providecommand{\sbrak}[1]{\ensuremath{{}\left[#1\right]}}
\providecommand{\lsbrak}[1]{\ensuremath{{}\left[#1\right.}}
\providecommand{\rsbrak}[1]{\ensuremath{{}\left.#1\right]}}
\providecommand{\brak}[1]{\ensuremath{\left(#1\right)}}
\providecommand{\lbrak}[1]{\ensuremath{\left(#1\right.}}
\providecommand{\rbrak}[1]{\ensuremath{\left.#1\right)}}
\providecommand{\cbrak}[1]{\ensuremath{\left\{#1\right\}}}
\providecommand{\lcbrak}[1]{\ensuremath{\left\{#1\right.}}
\newcommand*{\comb}[1][-1mu]{\permcomb[#1]{C}}
\renewcommand{\thetable}{\arabic{table}}
\providecommand{\rcbrak}[1]{\ensuremath{\left.#1\right\}}}

\textbf{Example 7-8:}Using the equation,
\begin{equation}
    \phi_z\brak{w} = E\cbrak{e^{jw(x_1+x_2+....x_n)}} = \phi_1\brak{w}....\phi_n\brak{w} \label{Eq 1}
\end{equation}
where $x_i$ are independent and $\phi_i\brak{w}$ is the characteristic function of $x_i$. \\
Prove:
\begin{enumerate}[label = (\alph*)]
    \item Bernoulli Trials Fundamental theorem
    \item Poisson Theorem
\end{enumerate}
\solution
\begin{enumerate}[label = (\alph*)]
    \item Bernoulli Trials:
    
    Consider the random variable $x_i$ as follows: $x_i = 1$ if head shows the at the ith trial and $x_i = 0$ otherwise. Thus,
    \begin{align}
        \pr{x_i = 1} &= \pr{h} = p \\
        \pr{x_i = 0} &= \pr{t} = q  \\
        \phi_i\brak{w} &= pe^{jw} + q
    \end{align}
    The random variable $z = x_1 + ....+x_n$ takes the values 0,1.....,n and $\cbrak{z = k}$ is the event \cbrak{\text{k heads in n tosses}}. Furthermore, 
    \begin{equation}
        \phi_z\brak{w} = E\cbrak{e^{jwz}} = \sum_{k=0}^{n}\pr{z = k}e^{jkw} \label{eq 5}
    \end{equation}
    The random variables $x_i$ are independent because $x_i$ depends only on the the outcomes of the ith trial.Hence,
    \begin{equation}
        \phi_i\brak{w} = \brak{pe^{jw} + q}^n = \sum_{k=0}^{n}\comb{n}{k}p^{k}e^{jkw}q^{n-k}
    \end{equation}
    Comparing with \eqref{eq 5}, we get,
    \begin{equation}
        \pr{z = k} = \pr{\text{k heads}} = \comb{n}{k}p^{k}q^{n-k}
    \end{equation}
    Hence Proved.
    \item Poisson Theorem:
    
    We will show that if $p << 1$, then
    \begin{equation}
        \pr{z = k} = \frac{e^{-np}\brak{np}^k}{k!}.
    \end{equation}
    We will establish a more general result. Suppose the random variables $x_i$ are independent and each takes values 1 and 0 with respective probabilities $p_i$ and $q_i = 1 - p_i$. If $p_i << 1$ then,
    \begin{align}
        e^{p_i\brak{e^{jw}-1}} &= 1 + p_i\brak{e^{jw} - 1} \\
                             &= p_ie^{jw} + q_i \\
                             &= \phi_i\brak{w}
    \end{align}
    With $z = x_1 + ....+x_n$, it follows from \eqref{Eq 1} that,
    \begin{equation}
        \phi_Z\brak{w} = e^{p_1\brak{e^{jw}-1}}...e^{p_n\brak{e^{jw}-1}} = e^{a\brak{e^{jw}-1}}
    \end{equation}
    where $a = p_1 + ....+p_n$. This leads to the conclusion that the random variable z is approximately Poisson distributed with parameter a. It can be shown that the result is exact in the limit if
    
     $p_i \rightarrow 0$ and $p_1 + ....+p_n \rightarrow a$  as $n \rightarrow \infty$   
     
     Hence Proved
    
\end{enumerate}

\end{document}
