\documentclass{beamer}

\usetheme{CambridgeUS}

\usepackage{listings}
\usepackage{amssymb}
%\usepackage[cmex10]{amsmath}

\usepackage[export]{adjustbox}
\usepackage{bm}
\def\inputGnumericTable{} 

\usepackage[latin1]{inputenc}                                 
\usepackage{color}                                            
\usepackage{array}   
\usepackage{longtable}
\usepackage{enumitem}
\usepackage{calc}                                             
\usepackage{multirow}                                         
\usepackage{hhline}                                           
\usepackage{ifthen}  
\usepackage{mathtools}
\usepackage{tikz}
\usepackage{listings}
\usepackage{color}                                            %%
\usepackage{array}                                            %%
\usepackage{caption} 
\usepackage{graphicx}
\graphicspath{{images/}}
\captionsetup[table]{skip=3pt} 

 

\title{A1110 Assignment 12 }
\author{Tejal Kulkarni \\ CS21BTECH11058 \\\vspace*{20pt} Papoullis Text Book }

\begin{document}

%\newcommand{\solution}{\noindent \textbf{Solution: }}
\providecommand{\pr}[1]{\ensuremath{\Pr\left(#1\right)}}
\providecommand{\cdf}[2]{\ensuremath{\text{F}_{#1}\left(#2\right)}}
\providecommand{\qfunc}[1]{\ensuremath{Q\left(#1\right)}}
\providecommand{\sbrak}[1]{\ensuremath{{}\left[#1\right]}}
\providecommand{\lsbrak}[1]{\ensuremath{{}\left[#1\right.}}
\providecommand{\rsbrak}[1]{\ensuremath{{}\left.#1\right]}}
\providecommand{\brak}[1]{\ensuremath{\left(#1\right)}}
\providecommand{\lbrak}[1]{\ensuremath{\left(#1\right.}}
\providecommand{\rbrak}[1]{\ensuremath{\left.#1\right)}}
\providecommand{\cbrak}[1]{\ensuremath{\left\{#1\right\}}}
\providecommand{\lcbrak}[1]{\ensuremath{\left\{#1\right.}}
\providecommand{\rcbrak}[1]{\ensuremath{\left.#1\right\}}}
\newcommand*{\permcomb}[4][0mu]{{{}^{#3}\mkern#1#2_{#4}}}
\newcommand*{\perm}[1][-3mu]{\permcomb[#1]{P}}
\newcommand*{\comb}[1][-1mu]{\permcomb[#1]{C}}
\renewcommand{\thetable}{\arabic{table}} 

\begin{frame}
    \titlepage
\end{frame}

\begin{frame}{Outline}
  \tableofcontents
\end{frame}

\section{Question}
\begin{frame}{Question}
\textbf{Problem 12-19:}Find the maximum entropy estimate $S_{MEM}\brak{w}$ and the line-spectral estimate of a process x\sbrak{n} if 
\begin{align}
    R\sbrak{0} &= 13 \\
    R\sbrak{1} &= 5 \\
    R\sbrak{2} &= 2
\end{align}
\end{frame}

\section{Solution}
\begin{frame}{Solution}
By Levinson's algorithm,
\begin{align}
    a_1^1 = K_1 &= \frac{R\sbrak{1}}{R\sbrak{0}} \\
    P_1 &= \brak{1 - K_1^2}R\sbrak{0} \\
    P_N &= \brak{1 - K_N^2}P_{N-1} \label{Eq:6}
\end{align}
Hence,
\begin{align}
    P_0 &= R\sbrak{0} = 13 \\
    a_1^1 &= K_1 = \frac{5}{13} \\
    P_1 &= \frac{144}{13}
\end{align}
\end{frame}

\begin{frame}{}
Also,
\begin{align}
    P_{N-1}K_N &= R\sbrak{N} - \sum_{k=1}^{N-1}a_k^{N-1}R\sbrak{N-1} \\
    a_N^N &= K_N \\ 
    a_k^N &= a_k^{N-1} - K_Na_{N-k}^{N-1} \label{Eq:12}
\end{align}
where $1\leq k \leq N-1$

This gives,
\begin{align}
    P_1K_2 &= R\sbrak{2} - a_1^1R\sbrak{1} \\
    \implies K_2 &= \frac{1}{144} \\
    \implies a_2^2 &= \frac{1}{144}
\end{align}
\end{frame}

\begin{frame}{}
 By \eqref{Eq:12},
\begin{align}
    a_1^2 &= a_1^1 - K_2a_1^1 \\
    \implies a_1^2 &= \frac{55}{144}
\end{align}
By \eqref{Eq:6},
\begin{align}
    P_2 &= \brak{1 - K_2^2}P_1 \\
    \implies P_2 &= \frac{1595}{144}
\end{align}
Hence,
\begin{equation}
   S_{MEM}\brak{w} = \frac{1595\times 144}{|144-55e^{-jwT}-e^{-j2wT}|^2} 
\end{equation}   
\end{frame}

\begin{frame}{}
 The Yule-Walker's equations are given by,
\begin{align}
    R\sbrak{0} + a_1R\sbrak{1} + ...+ a_NR\sbrak{N} &= P_N \\
    R\sbrak{1} + a_1R\sbrak{0} + ...+ a_NR\sbrak{N-1} &= 0 \\
    ......................................................... \\
    R\sbrak{N} + a_1R\sbrak{N-1} + ...+ a_NR\sbrak{0} &= 0 
\end{align}
The correlation matrix $D_{N+1}$ is given by,
\[
  D_{N+1} =
  \left[ {\begin{array}{cccc}
    R_{yy}\sbrak{0}-q & R_{yy}\sbrak{1} & \cdots & R_{yy}\sbrak{N}\\
    R_{yy}\sbrak{1} & R_{yy}\sbrak{0}-q & \cdots & R_{yy}\sbrak{N-1}\\
    \cdots & \cdots & \cdots & \cdots\\
    R_{yy}\sbrak{N} & R_{yy}\sbrak{N-1} & \cdots & R_{yy}\sbrak{0}-q\\
  \end{array} } \right]
\]   
\end{frame}

\begin{frame}{}
    To find $q_0$,
\begin{equation}
    \begin{vmatrix}
   13-q & 5 & 2 \\
   5 & 13-q & 5 \\
   2 & 5 & 13-q
\end{vmatrix} 
= 0
\end{equation}
This gives $q_0 = 14 - \sqrt{51} \simeq 6.86$.  
Inserting the modified data in the Yule Walker's equations we obtain,
\begin{align}
    a_1^2 \simeq 4.07 \\
    a_2^2 = -1
\end{align}
\end{frame}

\begin{frame}{}
    Also,
\begin{align}
    E_N\brak{z} &= 1 - a_1^Nz^{-1} - ... - a_N^Nz^{-N} \\ \label{Eq:28}
    \implies E_2\brak{z} &= 1 - 4.07z^{-1} + z^{-2} \\
     z_{1,2} &= e^{\pm j0.62}
\end{align}
By \eqref{Eq:28} we get,
\begin{equation}
    \varepsilon\sbrak{n} = x\sbrak{n} - a_1^Nx\sbrak{n-1} - ... - a_N^Nx\sbrak{n-N}
\end{equation}
Solving this we get,
\begin{align}
    R_L\sbrak{m} &= 6.86 \times \delta\sbrak{m} + 3.07 \cos{0.62m} \\
    S_L \brak{m} &= 6.86 + \frac{2\pi}{T}\times 3.07\sbrak{\delta\brak{w-0.62} + \delta\brak{w + 0.62}}
\end{align}
\end{frame}

\end{document}