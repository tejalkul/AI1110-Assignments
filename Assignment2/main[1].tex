\documentclass[journal,12pt,two column]{IEEEtran}
\usepackage[utf8]{inputenc}
\usepackage{amssymb}
\usepackage[cmex10]{amsmath}
\usepackage{amsthm}
\usepackage[export]{adjustbox}
\usepackage{bm}
\usepackage{longtable}
\usepackage{enumitem}
\usepackage{mathtools}
\usepackage{tikz}
\usepackage{listings}
\usepackage{color}                                            %%
\usepackage{array}                                            %%

\title{A1110 Assignment 2 }
\author{Tejal Kulkarni \\ CS21BTECH11058 \\\vspace*{20pt} April 2022 \\ ICSE 12th 2018 }


\begin{document}
\providecommand{\sbrak}[1]{\ensuremath{{}\left[#1\right]}}
\newcommand{\solution}{\noindent \textbf{Solution: }}
\maketitle

\textbf{ Q21(b):} A manufacturer's marginal cost function is $\dfrac{500}{\sqrt{2x+25}}$. Find the cost involved to increase production from 100 units to 300 units.\\
\solution
\begin{table}[ht!] 
\caption{\textbf{Table with input and output variables, their symbols, their formulae and values:}}
\label{table:1}
\begin{tabular}{|p{3.5cm}|c|c|c|}
       \hline
     \textbf{Description}  & \textbf{Symbol} &  \textbf{Formula}  &  \textbf{Value} \\
       \hline
       \hline
       Marked Price & MP & - & Rs.45000 \\  
       \hline
       Discount for shopkeeper & d1 & - & 10\% \\
       \hline
       Discount amount for shopkeeper & disc1 & $\text{MP}\times\dfrac{d1}{100}$ & Rs.4500\\
       \hline
       Selling Price for shopkeeper & SP1 & $\text{MP} -  \text{disc1}$ & Rs.40500 \\
       \hline
       Sales Tax & s & - & 12\% \\
       \hline
       Tax amount for shopkeeper & t1 & $\text{SP1}\times\dfrac{s}{100}$ & Rs.4860\\
       \hline
       Discount for customer & d2 & - & 5\% \\
       \hline
       Discount amount for customer & disc2 & $\text{MP}\times\dfrac{d2}{100}$ & Rs.2250 \\
       \hline
       Selling Price for customer & SP2 & $\text{MP} - \text{disc2}$ & Rs.42750 \\
       \hline
       Tax amount for customer & t2 & $\text{SP2}\times\dfrac{s}{100}$ & Rs.5130\\
       \hline
       VAT paid by shopkeeper to govt. & V & $\text{t2} - \text{t1}$ & ?\\
       \hline
       Total Amount paid by customer & T & $\text{SP2} + \text{t2}$ & ?\\
       \hline
       
    \end{tabular}

\end{table}

From the table we have cost involved to increase production from a units to b units
\begin{equation}
\text{C} = \int_{a}^{b} F(x) \,dx    
\end{equation}
where F(x)dx is the cost involved in changing production by dx units. Hence for a to b we integrate F(x) from a to b units
\begin{align}
\implies \text{C} &= \int_{a}^{b} \frac{500}{\sqrt{2x+25}} \,dx \\
                   &= 500 \int_{a}^{b} \frac{1}{\sqrt{2x+25}} \,dx  
\end{align}
\begin{equation}
\text{Let},\hspace{0.2cm} 2\text{x} + 25 = \text{t}, 2\text{dx} = \text{dt}    
\end{equation}
\begin{align}
\implies \text{C} &= 500 \int_{2a+25}^{2b+25} \frac{1}{\sqrt{\text{t}}} \,\frac{\text dt}{2} \\
&= \frac{500}{2} \int_{2a+25}^{2b+25} \frac{1}{\sqrt{\text{t}}} \,\text{dt} \\
&= \frac{500}{2} \sbrak{2\sqrt{\text{t}}}_{2a+25}^{2b+25} \\
&= 500 \sbrak{\sqrt{\text{t}}}_{2a+25}^{2b+25} \\
&= 500(\sqrt{2b + 25} - \sqrt{2a + 25})
\end{align}
\begin{equation} 
\text{Now},\hspace{0.2cm}\text{a} = 100 \hspace{0.2cm} \text{units} , \text{b} = 300 \hspace{0.2cm} \text{units}
\end{equation}
\begin{align}
\implies \text{C} &= 500(\sqrt{2\times 300+25}-\sqrt{2\times 100+25})\\
                  &= 500\times(\sqrt{625} - \sqrt{225})\\
                  &= 500\times(25 - 15) = 5000 \\
\implies \text{C} &= \text{Rs.}5000
\end{align}
Hence cost involved to increase production from 100 units to 300 units is \fbox{Rs.5000}.

\end{document}
