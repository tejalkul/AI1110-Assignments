\documentclass[journal,12pt,two column]{IEEEtran}
%\usepackage{setspace}
\usepackage{listings}
\usepackage{amssymb}
\usepackage[cmex10]{amsmath}
\usepackage{amsthm}
\usepackage[export]{adjustbox}
\usepackage{bm}
\def\inputGnumericTable{} 

\usepackage[latin1]{inputenc}                                 
\usepackage{color}                                            
\usepackage{array} 
\usepackage{longtable} 
\usepackage{calc}                                             
\usepackage{multirow}                                         
\usepackage{hhline}                                           
\usepackage{ifthen}  
\usepackage{mathtools}
\usepackage{tikz}
\usepackage{listings}
\usepackage{color}                                            %%
\usepackage{array}                                            %%
\usepackage{caption} 
\usepackage{graphicx}
\graphicspath{{images/}}
\captionsetup[table]{skip=3pt} 

\title{A1110 Assignment 3 }
\author{Tejal Kulkarni \\ CS21BTECH11058 \\\vspace*{20pt} April 2022 \\ CBSE Probability Grade 9 }
\begin{document}
\newcommand{\solution}{\noindent \textbf{Solution: }}
\providecommand{\pr}[1]{\ensuremath{\Pr\left(#1\right)}}
\providecommand{\qfunc}[1]{\ensuremath{Q\left(#1\right)}}
\providecommand{\sbrak}[1]{\ensuremath{{}\left[#1\right]}}
\providecommand{\lsbrak}[1]{\ensuremath{{}\left[#1\right.}}
\providecommand{\rsbrak}[1]{\ensuremath{{}\left.#1\right]}}
\providecommand{\brak}[1]{\ensuremath{\left(#1\right)}}
\providecommand{\lbrak}[1]{\ensuremath{\left(#1\right.}}
\providecommand{\rbrak}[1]{\ensuremath{\left.#1\right)}}
\providecommand{\cbrak}[1]{\ensuremath{\left\{#1\right\}}}
\providecommand{\lcbrak}[1]{\ensuremath{\left\{#1\right.}}
\providecommand{\rcbrak}[1]{\ensuremath{\left.#1\right\}}}
\newcommand*{\permcomb}[4][0mu]{{{}^{#3}\mkern#1#2_{#4}}}
\newcommand*{\perm}[1][-3mu]{\permcomb[#1]{P}}
\newcommand*{\comb}[1][-1mu]{\permcomb[#1]{C}}
\renewcommand{\thetable}{\arabic{table}} 

\maketitle

\textbf{Exercise 15.1 Q4:} Three coins are tossed simultaneously 200 times with the following frequencies of different items:
\begin{table}[ht!]
    \centering
    \input{table1}
    \caption{}
    \label{Table 1}
\end{table}

If the three coins are simultaneously tossed again, compute the probability of 2 heads coming up. 
\solution
Let the random variable $X \in \cbrak{0,1,2,3}$ denote the number of heads in the coin-tossing experiment. Now, 
\begin{equation}
   \pr{X = i} = \dfrac{n(X = i)}{\sum_{i=0}^{3} n(X = i) }
\end{equation}
where $i \in \cbrak{0,1,2,3}$ and n(X = i) is the frequency of getting i heads. Also,
\begin{align}
&\text{Number of times 3 coins were tossed} = 200\\ 
&\implies  \sum_{i=0}^{3} n(X = i) = 200
\end{align}
And from Table \ref{Table 1}, 
\begin{align}
                n(X = 2) &= 72 \\
\therefore    \pr{X = 2} &= \frac{72}{200} \\
                         &= \frac{36}{100} = 0.36 
\end{align}
Hence, the probability of 2 heads coming up is \fbox{0.36}.

We have,
\begin{align}
\pr{X = 0} &= \frac{28}{200} = 0.14 \\
\pr{X = 1} &= \frac{77}{200} = 0.385 \\
\pr{X = 2} &= \frac{72}{200} = 0.36 \\
\pr{X = 3} &= \frac{23}{200} = 0.115 
\end{align}
\begin{figure}[ht!]
     \centering
     \includegraphics[width = \columnwidth]{PMF1.png}
     \caption{Plot of PMF using above data}
     \label{fig:Figure 1}
\end{figure}

\textbf{Now considering fair coins:}
Let probability of getting a head be a success and equal to p and probability of getting a tail be a failure and equal to q where $p+q = 1$. We can express this as a binomial distribution
\begin{equation}
   \sum_{i=0}^{n} \pr{X = i} =  \sum_{i=0}^{n} \comb{n}{i}(\text{p})^i\left(1-\text{p}\right)^{n-i}
\end{equation}
where $n = 3$ for 3 coins. Therefore,
\begin{equation}
\pr{X = i} = \comb{3}{i} (\text{p})^i (\text{q})^{3-i}    
\end{equation}
For fair coins, 
\begin{align}
   \text{p} &= \frac{1}{2}\\
  \therefore  \text{q} &= \frac{1}{2}
\end{align}
Therefore,
\begin{align}
    \pr{X = 0} &= \comb{3}{0}\brak{\frac{1}{2}}^0\brak{\frac{1}{2}}^3 = \frac{1}{8}\\
    \pr{X = 1} &= \comb{3}{1}\brak{\frac{1}{2}}^1\brak{\frac{1}{2}}^2 = \frac{3}{8}\\
    \pr{X = 2} &= \comb{3}{2}\brak{\frac{1}{2}}^2\brak{\frac{1}{2}}^1 = \frac{3}{8}\\
    \pr{X = 3} &= \comb{3}{3}\brak{\frac{1}{2}}^3\brak{\frac{1}{2}}^0 = \frac{1}{8}
    \end{align}
\begin{figure}[!ht]
     \centering
     \includegraphics[width = \columnwidth]{PMF2.png}
     \caption{Comparison of theoretical and practical PMF plots}
     \label{fig:Figure 2}
\end{figure}

\end{document}
