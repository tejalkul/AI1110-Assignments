\documentclass[journal,12pt,two column]{IEEEtran}
%\usepackage{setspace}
\usepackage{listings}
\usepackage{amssymb}
\usepackage[cmex10]{amsmath}
\usepackage{amsthm}
\usepackage[export]{adjustbox}
\usepackage{bm}
\def\inputGnumericTable{} 

\usepackage[latin1]{inputenc}                                 
\usepackage{color}                                            
\usepackage{array}                                                \usepackage{longtable}                                        
\usepackage{calc}                                             
\usepackage{multirow}                                         
\usepackage{hhline}                                           
\usepackage{ifthen}  
\usepackage{mathtools}
\usepackage{tikz}
\usepackage{listings}
\usepackage{color}                                            %%
\usepackage{array}                                            %%
\usepackage{caption} 
\usepackage{graphicx}
\graphicspath{{images/}}
\captionsetup[table]{skip=3pt} 

\title{A1110 Assignment 4 }
\author{Tejal Kulkarni \\ CS21BTECH11058 \\\vspace*{20pt} April 2022 \\ CBSE Probability Grade 10 }

\begin{document}
\newcommand{\solution}{\noindent \textbf{Solution: }}
\providecommand{\pr}[1]{\ensuremath{\Pr\left(#1\right)}}
\providecommand{\qfunc}[1]{\ensuremath{Q\left(#1\right)}}
\providecommand{\sbrak}[1]{\ensuremath{{}\left[#1\right]}}
\providecommand{\lsbrak}[1]{\ensuremath{{}\left[#1\right.}}
\providecommand{\rsbrak}[1]{\ensuremath{{}\left.#1\right]}}
\providecommand{\brak}[1]{\ensuremath{\left(#1\right)}}
\providecommand{\lbrak}[1]{\ensuremath{\left(#1\right.}}
\providecommand{\rbrak}[1]{\ensuremath{\left.#1\right)}}
\providecommand{\cbrak}[1]{\ensuremath{\left\{#1\right\}}}
\providecommand{\lcbrak}[1]{\ensuremath{\left\{#1\right.}}
\providecommand{\rcbrak}[1]{\ensuremath{\left.#1\right\}}}
\newcommand*{\permcomb}[4][0mu]{{{}^{#3}\mkern#1#2_{#4}}}
\newcommand*{\perm}[1][-3mu]{\permcomb[#1]{P}}
\newcommand*{\comb}[1][-1mu]{\permcomb[#1]{C}}
\renewcommand{\thetable}{\arabic{table}} 

\maketitle

\textbf{Exercise 15.2 Q4:} A box contains 12 balls out of which x are black.If one ball is drawn at random from the box, what is the probability that it will be a black ball?

If 6 more black balls are put in the box, the probability of drawing a black ball is now double of what it was before. Find x.

\solution
Let the random variable $X \in \cbrak{0,1}$ denote whether a ball drawn out of the box is black or not.
\begin{table}[ht!]
    \centering
    \input{table2}
    \caption{}
    \label{Table 1}
\end{table}
\begin{align}
 \pr{X = 0} &= \dfrac{\text{Number of black balls}}{\text{Total Balls}}\\
            &= \frac{x}{12}
\end{align}
\begin{equation}
    \therefore \text{Probability that ball drawn is black} = \frac{x}{12}
\end{equation}
Now, 6 more black balls are added
\begin{align}
    \implies \text{Number of black balls} &= x + 6 \\
    \text{Total balls} = 12 + 6 &= 18
\end{align}
\begin{equation}
   \therefore \pr{X = 0} = \frac{x + 6}{18}
\end{equation}
Given,
\begin{align}
    2\times\frac{x}{12} &= \frac{x + 6}{18}\\
    \implies \frac{x}{6} &= \frac{x + 6}{18}\\
     x &= \frac{x + 6}{3}\\
     3x &= x + 6 \\
     2x &= 6 \\
     \implies x &= 3
\end{align}
\begin{figure}[!ht]
     \centering
     \includegraphics[width = \columnwidth]{Verification.png}
     \caption{Verification of the number of black balls}
     \label{fig:Figure 2}
\end{figure}

Hence,
\begin{align}
    \pr{X = 0}\text{(initial)} &= \frac{3}{12} = \frac{1}{4}\\
    \pr{X = 0}\text{(final)} &= \frac{9}{18} = \frac{1}{2}
\end{align}

\end{document}
