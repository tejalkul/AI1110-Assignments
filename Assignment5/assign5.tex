\documentclass[journal,12pt,two column]{IEEEtran}
\usepackage{setspace}
\usepackage{listings}
\usepackage{amssymb}
\usepackage[cmex10]{amsmath}
\usepackage{amsthm}
\usepackage[export]{adjustbox}
\usepackage{bm}
\def\inputGnumericTable{} 

\usepackage[latin1]{inputenc}                                 
\usepackage{color}                                            
\usepackage{array}   
\usepackage{longtable}
\usepackage{enumitem}
\usepackage{calc}                                             
\usepackage{multirow}                                         
\usepackage{hhline}                                           
\usepackage{ifthen}  
\usepackage{mathtools}
\usepackage{tikz}
\usepackage{listings}
\usepackage{color}                                            %%
\usepackage{array}                                            %%
\usepackage{caption} 
\usepackage{graphicx}
\graphicspath{{images/}}
\captionsetup[table]{skip=3pt} 

\title{A1110 Assignment 5 }
\author{Tejal Kulkarni \\ CS21BTECH11058 \\\vspace*{20pt} May 2022 \\ CBSE Probability Grade 11 }

\begin{document}
\newcommand{\solution}{\noindent \textbf{Solution: }}
\providecommand{\pr}[1]{\ensuremath{\Pr\left(#1\right)}}
\providecommand{\qfunc}[1]{\ensuremath{Q\left(#1\right)}}
\providecommand{\sbrak}[1]{\ensuremath{{}\left[#1\right]}}
\providecommand{\lsbrak}[1]{\ensuremath{{}\left[#1\right.}}
\providecommand{\rsbrak}[1]{\ensuremath{{}\left.#1\right]}}
\providecommand{\brak}[1]{\ensuremath{\left(#1\right)}}
\providecommand{\lbrak}[1]{\ensuremath{\left(#1\right.}}
\providecommand{\rbrak}[1]{\ensuremath{\left.#1\right)}}
\providecommand{\cbrak}[1]{\ensuremath{\left\{#1\right\}}}
\providecommand{\lcbrak}[1]{\ensuremath{\left\{#1\right.}}
\providecommand{\rcbrak}[1]{\ensuremath{\left.#1\right\}}}
\newcommand*{\permcomb}[4][0mu]{{{}^{#3}\mkern#1#2_{#4}}}
\newcommand*{\perm}[1][-3mu]{\permcomb[#1]{P}}
\newcommand*{\comb}[1][-1mu]{\permcomb[#1]{C}}
\renewcommand{\thetable}{\arabic{table}} 

\maketitle

\textbf{Exercise 16.3 Q8:} Three coins are tossed once. Find the probability of getting:
\begin{enumerate}[label = (\roman*)]
    \item 3 heads 
    \item 2 heads 
    \item atleast 2 heads 
    \item atmost 2 heads 
    \item no head 
    \item 3 tails 
    \item exactly two tails 
    \item no tail 
    \item atmost two tails
\end{enumerate}
\solution
Let the Bernoulli random variable $X \in \cbrak{0,1}$ where $X = 0$ denotes occurrence of head(success) and $X = 1$ denotes occurrence of tail(failure) for a single coin toss.For a fair coin,
\begin{align}
    \pr{X = 0} = p = \frac{1}{2} \label{1}\\
    \pr{X = 1} = q = \frac{1}{2} \label{2}
\end{align}
Let the Binomial random variable $Y \in \cbrak{0,1,2,3}$ denote the number of heads. We can express this as a binomial distribution,
\begin{equation}
    \pr{Y = k} = \comb{n}{k} (\text{p})^k (\text{q})^{n-k}  
\end{equation}
where $k \in \cbrak{0,1,2,3}$ and n = 3 for 3 coins. By \eqref{1} and \eqref{2},
\begin{equation}
    \pr{Y = k} = \comb{3}{k} \brak{\frac{1}{2}}^k \brak{\frac{1}{2}}^{3-k}  
\end{equation}
\begin{table}[ht!]
    \centering
    \input{table3}
    \caption{}
    \label{Table 1}
\end{table}
From Table \ref{Table 1},
\begin{enumerate}[label = (\roman*)]
    \item 3 heads:
    \begin{align}
        \pr{Y = 3} &= \comb{3}{3} \brak{\frac{1}{2}}^3 \brak{\frac{1}{2}}^{0}\\
                   &= \frac{1}{8}
    \end{align}
    \item 2 heads:
    \begin{align}
        \pr{Y = 2} &= \comb{3}{2} \brak{\frac{1}{2}}^2 \brak{\frac{1}{2}}^{1}\\
                   &= \frac{3}{8}
    \end{align}
    \item atleast 2 heads:
    \begin{align}
        \pr{Y \ge 2} &= \pr{Y = 2} + \pr{Y = 3} \\
                   &=\comb{3}{3} \brak{\frac{1}{2}}^3 + \comb{3}{2} \brak{\frac{1}{2}}^3 \\
                   &= \frac{1}{2}
    \end{align}
    \item atmost 2 heads:
    
    \pr{Y \le 2}
    \begin{align}
         &= \pr{Y = 2} + \pr{Y = 1} + \pr{Y = 0} \\
         &= \comb{3}{2} \brak{\frac{1}{2}}^3 + \comb{3}{1} \brak{\frac{1}{2}}^3 + \comb{3}{0} \brak{\frac{1}{2}}^3\\
         &= \frac{7}{8}
    \end{align}
    \item no head:
    \begin{align}
        \pr{Y = 0} &= \comb{3}{0} \brak{\frac{1}{2}}^0 \brak{\frac{1}{2}}^{3}\\
                   &= \frac{1}{8}
    \end{align}
    \item 3 tails:
    \begin{align}
        \pr{Y = 0} &= \comb{3}{0} \brak{\frac{1}{2}}^0 \brak{\frac{1}{2}}^{3}\\
                   &= \frac{1}{8}
    \end{align}
    \item exactly two tails:
    \begin{align}
        \pr{Y = 1} &= \comb{3}{1} \brak{\frac{1}{2}}^1 \brak{\frac{1}{2}}^{2}\\
                   &= \frac{3}{8}
    \end{align}
    \item no tail
    \begin{align}
        \pr{Y = 3} &= \comb{3}{3} \brak{\frac{1}{2}}^3 \brak{\frac{1}{2}}^{0}\\
                   &= \frac{1}{8}
    \end{align}
    \item atmost two tails:
    
    \pr{Y \ge 1}
    \begin{align}
         &= \pr{Y = 1}+\pr{Y = 2}+\pr{Y = 3} \\
         &= \comb{3}{1} \brak{\frac{1}{2}}^3 + \comb{3}{2}\brak{\frac{1}{2}}^3 + \comb{3}{3} \brak{\frac{1}{2}}^3\\
         &= \frac{7}{8}
    \end{align}
\end{enumerate} 
\begin{figure}[ht!]
\centering
\includegraphics[width=\columnwidth]{PMF&CDF.png}
\caption{Plot of PMF(left) and CDF(right) } 
\label{Fig 1}
\end{figure}

\end{document}
