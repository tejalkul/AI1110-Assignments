\documentclass[journal,12pt,two column]{IEEEtran}
\usepackage{setspace}
\usepackage{listings}
\usepackage{amssymb}
\usepackage[cmex10]{amsmath}
\usepackage{amsthm}
\usepackage[export]{adjustbox}
\usepackage{bm}
\def\inputGnumericTable{} 

\usepackage[latin1]{inputenc}                                 
\usepackage{color}                                            
\usepackage{array}   
\usepackage{longtable}
\usepackage{enumitem}
\usepackage{calc}                                             
\usepackage{multirow}                                         
\usepackage{hhline}                                           
\usepackage{ifthen}  
\usepackage{mathtools}
\usepackage{tikz}
\usepackage{listings}
\usepackage{color}                                            %%
\usepackage{array}                                            %%
\usepackage{caption} 
\usepackage{graphicx}
\graphicspath{{images/}}
\captionsetup[table]{skip=3pt} 

\title{A1110 Assignment 6 }
\author{Tejal Kulkarni \\ CS21BTECH11058 \\\vspace*{20pt} May 2022 \\ CBSE Probability Grade 12 }

\begin{document}
\newcommand{\solution}{\noindent \textbf{Solution: }}
\providecommand{\pr}[1]{\ensuremath{\Pr\left(#1\right)}}
\providecommand{\qfunc}[1]{\ensuremath{Q\left(#1\right)}}
\providecommand{\sbrak}[1]{\ensuremath{{}\left[#1\right]}}
\providecommand{\lsbrak}[1]{\ensuremath{{}\left[#1\right.}}
\providecommand{\rsbrak}[1]{\ensuremath{{}\left.#1\right]}}
\providecommand{\brak}[1]{\ensuremath{\left(#1\right)}}
\providecommand{\lbrak}[1]{\ensuremath{\left(#1\right.}}
\providecommand{\rbrak}[1]{\ensuremath{\left.#1\right)}}
\providecommand{\cbrak}[1]{\ensuremath{\left\{#1\right\}}}
\providecommand{\lcbrak}[1]{\ensuremath{\left\{#1\right.}}
\providecommand{\rcbrak}[1]{\ensuremath{\left.#1\right\}}}
\newcommand*{\permcomb}[4][0mu]{{{}^{#3}\mkern#1#2_{#4}}}
\newcommand*{\perm}[1][-3mu]{\permcomb[#1]{P}}
\newcommand*{\comb}[1][-1mu]{\permcomb[#1]{C}}
\renewcommand{\thetable}{\arabic{table}} 

\maketitle

\textbf{Exercise 13.1 Q4:} Evaluate $\pr{A \cup B}$,if $2\pr{A} = \pr{B} = \dfrac{5}{13}$ and $\pr{A|B} = \dfrac{2}{5}$. 
\solution
Consider the random variable $X\in \cbrak{0,1}$, where $X = 0$ denotes the event A and $X=1$ denotes the event B. 
\begin{align}
 \implies \pr{X=0|X=1} &= \pr{A|B}  \\ 
 \therefore \pr{X=0|X=1} &= \frac{\pr{X=0,X=1}}{\pr{X=1}} 
\end{align}
\begin{multline}
    \implies \pr{X=0,X=1} \\
          =\pr{X=0|X=1}\times{\pr{X=1}} 
\end{multline}
\begin{equation}
    \pr{X=0, X=1} = \frac{2}{5}\times \frac{5}{13} = \frac{2}{13}
\end{equation}
Now,
\begin{multline}
    \pr{X=0 + X=1} = \pr{X = 0} + \pr{X = 1} \\ - \pr{X=0,X=1} 
\end{multline}
\begin{align}
  2\pr{X = 0} &= \frac{5}{13} \\
  \implies \pr{X = 0} &= \frac{5}{26} \\
  \pr{X = 1} &= \frac{5}{13} \\
  \pr{X=0,X=1} &= \frac{2}{13} \\
  \therefore \pr{X=0 + X=1} &= \frac{5}{26} + \frac{5}{13} - \frac{2}{13}\\
                              &= \frac{11}{26} \\
  \therefore \pr{A + B} &= \boxed{\frac{11}{26}}
\end{align}

    
\end{document}
