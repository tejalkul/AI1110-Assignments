\documentclass{beamer}

\usetheme{CambridgeUS}

\usepackage{listings}
\usepackage{amssymb}
%\usepackage[cmex10]{amsmath}

\usepackage[export]{adjustbox}
\usepackage{bm}
\def\inputGnumericTable{} 

\usepackage[latin1]{inputenc}                                 
\usepackage{color}                                            
\usepackage{array}   
\usepackage{longtable}
\usepackage{enumitem}
\usepackage{calc}                                             
\usepackage{multirow}                                         
\usepackage{hhline}                                           
\usepackage{ifthen}  
\usepackage{mathtools}
\usepackage{tikz}
\usepackage{listings}
\usepackage{color}                                            %%
\usepackage{array}                                            %%
\usepackage{caption} 
\usepackage{graphicx}
\graphicspath{{images/}}
\captionsetup[table]{skip=3pt} 

 

\title{A1110 Assignment 7 }
\author{Tejal Kulkarni \\ CS21BTECH11058 \\\vspace*{20pt} CBSE Probability Grade 12 }

\begin{document}

%\newcommand{\solution}{\noindent \textbf{Solution: }}
\providecommand{\pr}[1]{\ensuremath{\Pr\left(#1\right)}}
\providecommand{\qfunc}[1]{\ensuremath{Q\left(#1\right)}}
\providecommand{\sbrak}[1]{\ensuremath{{}\left[#1\right]}}
\providecommand{\lsbrak}[1]{\ensuremath{{}\left[#1\right.}}
\providecommand{\rsbrak}[1]{\ensuremath{{}\left.#1\right]}}
\providecommand{\brak}[1]{\ensuremath{\left(#1\right)}}
\providecommand{\lbrak}[1]{\ensuremath{\left(#1\right.}}
\providecommand{\rbrak}[1]{\ensuremath{\left.#1\right)}}
\providecommand{\cbrak}[1]{\ensuremath{\left\{#1\right\}}}
\providecommand{\lcbrak}[1]{\ensuremath{\left\{#1\right.}}
\providecommand{\rcbrak}[1]{\ensuremath{\left.#1\right\}}}
\newcommand*{\permcomb}[4][0mu]{{{}^{#3}\mkern#1#2_{#4}}}
\newcommand*{\perm}[1][-3mu]{\permcomb[#1]{P}}
\newcommand*{\comb}[1][-1mu]{\permcomb[#1]{C}}
\renewcommand{\thetable}{\arabic{table}} 


\begin{frame}
    \titlepage
\end{frame}

\begin{frame}{Outline}
  \tableofcontents
\end{frame}

\section{Question}

\begin{frame}{Question}
\textbf{Exercise 13.2 Q14:}Probability of solving specific problem independently by A and B are $\dfrac{1}{2}$ and $\dfrac{1}{3}$ respectively. If both try to solve the problem independently, find the probability that 
\begin{enumerate} [label = (\roman*)]
    \item the problem is solved
    \item exactly one of them solves the problem
\end{enumerate}
\end{frame}

\section{Given Information}

\begin{frame}{Given Information}
Solution: 
 
 Let  E and F be two events such that:
\begin{table}[ht!]
    \centering
    \input{table-4}
    \caption{}
    \label{Table1}
\end{table}
\end{frame} 

\section{Required Formulae}

\begin{frame}{Required Formulae}
E and F are independent events
\begin{align}
    \therefore \pr{EF} &= \pr{E}\pr{F} \\
               \pr{EF'} &= \pr{E}\pr{F'} \\
               \pr{E'F} &= \pr{E'}\pr{F} 
\end{align}
Also, for any event X we can write,
\begin{equation}
    \pr{X'} = 1 - \pr{X} \label{4}
\end{equation}
\end{frame}

\section{Solution of (i)}

\begin{frame}{Solution of (i)}
Now,
        
        \textbf{(i)} Probability that problem is solved = 
        \begin{align}
           \pr{E + F} &= \pr{E} + \pr{F} - \pr{EF} \\
            \pr{E+F} &= \pr{E} + \pr{F} - \pr{E}\pr{F}\\
            &= \frac{1}{2} + \frac{1}{3} - \frac{1}{2}\times\frac{1}{3}\\
            &= \frac{4}{6} \\
            &= \boxed{\frac{2}{3}}
          \end{align}
\end{frame}

\section{Solution of (ii)}

\begin{frame}{Solution of (ii)}
 \textbf{(ii)} Probability that exactly one of them solves the problem =
     \begin{equation}
      \pr{EF'} + \pr{E'F}  = \pr{E}\pr{F'} + \pr{E'}\pr{F}
    \end{equation}
    By \eqref{4},
    \begin{align}
     \pr{E}\pr{F'} + \pr{E'}\pr{F}   &= \pr{E}\brak{1-\pr{F}} + \brak{1-\pr{E}}\pr{F} \\
        &= \frac{1}{2}\times\brak{1-\frac{1}{3}} +\brak{1-\frac{1}{2}}\times \frac{1}{3} \\
        &= \frac{1}{2}\times\frac{2}{3} + \frac{1}{2}\times\frac{1}{3} \\
        &= \frac{1}{3} + \frac{1}{6} \\
        &= \frac{3}{6} = \boxed{\frac{1}{2}}
    \end{align}
    

\end{frame}

\end{document}