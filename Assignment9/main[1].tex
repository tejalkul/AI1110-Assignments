\documentclass[journal,12pt,two column]{IEEEtran}
\usepackage{setspace}
\usepackage{listings}
\usepackage{amssymb}
\usepackage[cmex10]{amsmath}
\usepackage{amsthm}
\usepackage[export]{adjustbox}
\usepackage{bm}
\def\inputGnumericTable{} 

\usepackage[latin1]{inputenc}                                 
\usepackage{color}                                            
\usepackage{array}   
\usepackage{longtable}
\usepackage{enumitem}
\usepackage{calc}                                             
\usepackage{multirow}                                         
\usepackage{hhline}                                           
\usepackage{ifthen}  
\usepackage{mathtools}
\usepackage{tikz}
\usepackage{listings}
\usepackage{color}                                            %%
\usepackage{array}                                            %%
\usepackage{caption} 
\usepackage{graphicx}
\graphicspath{{images/}}
%\captionsetup[table]{skip=3pt} 

\title{A1110 Assignment 9 }
\author{Tejal Kulkarni \\ CS21BTECH11058 \\\vspace*{20pt} May 2022 \\  Papoullis Text Book }

\begin{document}
\maketitle

\newcommand{\solution}{\noindent \textbf{Solution: }}
\providecommand{\pr}[1]{\ensuremath{\Pr\left(#1\right)}}
\providecommand{\cdf}[2]{\ensuremath{\text{F}_{#1}\left(#2\right)}}
\providecommand{\qfunc}[1]{\ensuremath{Q\left(#1\right)}}
\providecommand{\sbrak}[1]{\ensuremath{{}\left[#1\right]}}
\providecommand{\lsbrak}[1]{\ensuremath{{}\left[#1\right.}}
\providecommand{\rsbrak}[1]{\ensuremath{{}\left.#1\right]}}
\providecommand{\brak}[1]{\ensuremath{\left(#1\right)}}
\providecommand{\lbrak}[1]{\ensuremath{\left(#1\right.}}
\providecommand{\rbrak}[1]{\ensuremath{\left.#1\right)}}
\providecommand{\cbrak}[1]{\ensuremath{\left\{#1\right\}}}
\providecommand{\lcbrak}[1]{\ensuremath{\left\{#1\right.}}
\providecommand{\rcbrak}[1]{\ensuremath{\left.#1\right\}}}
\newcommand*{\permcomb}[4][0mu]{{{}^{#3}\mkern#1#2_{#4}}}
\newcommand*{\perm}[1][-3mu]{\permcomb[#1]{P}}
\newcommand*{\comb}[1][-1mu]{\permcomb[#1]{C}}
\renewcommand{\thetable}{\arabic{table}}

\textbf{Example 5-17:}If x takes the values 1, 2, ... 6 with probability 1/6, then find E(x).

\solution
For discrete random variables, 
\begin{equation}
    E(x) = \sum_{i} p_ix_i   
\end{equation}
Given,
\begin{equation}
    p_i = P(X = x_i) = \frac{1}{6}
\end{equation}
where $x_i\in \cbrak{0,1,2,3,4,5,6}$
Therefore,
\begin{align}
    E(x) &= \frac{1}{6}\sum_{i} x_i \\
         &= \frac{1}{6}\brak{1+2+3+4+5+6} \\
         &= \frac{21}{6} \\
         &= \frac{7}{2} \\
         &= \boxed{3.50}
\end{align}


\end{document}
